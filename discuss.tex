The relational data model has served as the theoretical foundation for the majority of mainstream  database systems for over forty years and has been shown to be the most cogent and principled approach for representing and manipulating data of arbitrary complexity \cite{date_sql_2011}.
Yet, relational organization of data in science labs is still uncommon. 
This disconnect is partly due to SQL's  position as the only wide-spread implementation of the relational model. 
SQL and its dialects have deviated substantially from the simplicity of the relational data model and have overgrown with extraneous complexity.

Object-relational mappers (ORM) are software tools that map objects in computer memory to persistent storage such as relational databases.
Since DataJoint constructs objects with persistently stored data, it can be classified as an ORM.
Several other object-relational mappers are available for Python: SQLAlchemy, Django ORM, Peewee, PonyORM, SQLObject.
They take a similar approach of converting objects and idioms of the host programming language into SQL queries for processing by the database server.
We are not aware of any ORM tools for MATLAB besides DataJoint.

DataJoint addresses other needs than ORMs: it is specifically designed for providing a robust and intuitive data model for scientific data processing chains.
As such, it does not attempt to simply mirror the features and capabilities of SQL.
Instead, DataJoint imposes constraints and conventions to achieve the expressive power and simplicity of queries by strict adherence to the relational data model.
In science, both the structure of the data and the queries evolve frequently. 
A simple, sound data model is of greater importance than in other database applications.

Some of the distinct constraints imposed by DataJoint include the following:
\begin{enumerate}

\item
All data are represented as proper relations with a primary key and uniquely named attributes. 
This applies to base and derived relations. 
Relational operators follow consistent rules for determining the primary key of its result. 
As a result, DataJoint's operators are algebraically closed, allowing building complex expressions from simpler expressions.

\item
Data in base relations are updated only by inserting or deleting entire tuples: updates of attribute values are not supported.
As discussed in the text, this limitation is necessary because referential constraints (foreign keys) enforce data dependencies only between tuples and not between individual attribute values. 

\item 
Dependencies between base relations are acyclic, \emph{i.e.}\ they cannot form loops. 
This restriction simplifies data definition but, perhaps counter-intuitively, does not prevent specifying arbitrary relationships between data elements, including directed graphs with cyclic dependencies, for example.

\item
DataJoint limits relational operators to enforce clarity.
For example, the projection operator (see Table \ref{algebra} and online documentation) does not allow projecting out the primary key attributes.
Consequently, the resulting relation has the same number of tuples as the original relation and every tuple is unique.
If the user does intend to derive a relation with a different primary key, she must explicitly declare a base relation with this primary key and use it to formulate the proper query.
In practice, this is not a real limitation but a specific prescription of how data must be defined and manipulated in a uniform and explicit manner.

\item
Foreign keys always link identically named attributes in both relations.
This convention simplifies the specification of dependencies and of relational operators.
For example, a single join operator in DataJoint can perform the same work as the multiple forms and parameterizations of the JOIN operators in SQL.
This convention is particularly important in DataJoint because it allows direct logical linking of relations separated by many intermediate dependencies.
In a large schema, this convention may lead to long composite primary keys low in the dependency hierarchy, but these are efficiently handled by MySQL's storage engines.

\end{enumerate}

DataJoint's restricted relational data model represents a conceptual shift in database interactions: In SQL queries, users explicitly enumerate and match individual attributes.
In contrast, DataJoint users formulate dependencies and queries at the level of entire relations.
As a result, DataJoint's fast, intuitive, and expressive data definition and manipulation languages enable scientists to flexibly adapt their data processing chains to evolving demands.
